\apendice{Documentación de usuario}

\section{Introducción}

En este apartado se van a detallar los requisitos necesarios para poder ejecutar la aplicación Android, así como la web de administración. A parte de esto, se especifica los pasos necesarios para la instalación y el funcionamiento de los mismos.

\section{Requisitos de usuarios}

\subsubsection{Aplicación Android}

Los requisitos para poder ejecutar la aplicación en Android son:

\begin{itemize}
	\tightlist
	\item 
	Dispositivo Android con versión 4.4 o superior.
	\item 
	Acceso a Internet (WiFi o datos móviles).
\end{itemize}

\subsubsection{Web de administración}

Los requisitos necesarios para poder utilizar la web de administración son:

\begin{itemize}
	\tightlist
	\item 
	Dispositivo con posibilidad de utilizar un navegador web.
	\item 
	Navegador web con JavaScript compatible y activado.
	\item 
	Acceso a Internet.
\end{itemize}

\section{Instalación}

\subsubsection{Aplicación Android}

Para poder instalar la aplicación APK, primero tenemos que activar la opción \textit{Orígenes desconocidos} en el dispositivo. Para ello accedemos a los ajustes del dispositivo, accedemos a la opción \textit{Seguridad} y activamos la opción \textit{Orígenes desconocidos}.

\imagenPeque{CapturaSeguridad.png}{Opciones de seguridad Android}

A continuación procedemos a la instalación de la aplicación, para ello primero tenemos que copiar el fichero APK a nuestro dispositivo. Una vez que termine la copia del fichero, desde nuestro dispositivo abrimos el fichero con un explorador de archivos. En la ventana que nos aparece, le damos a la opción \textit{Instalar}.

\imagenPeque{CapturaInstalacion.png}{Instalación UBUassistant}

Con estos pasos, la aplicación ya está instalada en el dispositivo Android.

\subsection{Web de administración}

Para poder utilizar la web de administración solamente es necesario tener instalado en el dispositivo en el que se vaya a utilizar un navegador web, ya que la aplicación de administración está alojado en un servidor.

Para acceder a la página de administración en el navegador hay que introducir la siguiente URL:

\begin{center}
	\textit{\textbf{http://IP\_SERVIDOR:8080/UBUassistant\_admin}}
\end{center}

\section{Manual del usuario}

En este manual se trata de guiar al usuario en el funcionamiento del asistente virtual con la aplicación Android, así como en la aplicación de administración web.

\subsection{Aplicación Android}

Lo primero que vemos al iniciar la aplicación es una ventana en la que aparecen el logo de la aplicación, junto al de la Universidad de Burgos. Lo siguiente que aparece es la interfaz del asistente virtual, la cual se asemeja a diferentes aplicaciones de mensajería para dispositivos móviles.

\imagenMediana{CapturaInterfaz.png}{Interfaz aplicación Android}

\newpage

En el caso de que los servidores de la aplicación no se encuentren disponibles, la aplicación se dirigrá automáticamente a una ventana en la que se muestra un mensaje de advertencia.

\imagenMediana{CapturaError.png}{Aviso de error de conexión}

En la parte inferior, encontramos una barra, la cual encontramos un cuadro donde podemos introducir el texto para interactuar con el asistente virtual. También nos encontramos con un botón para poder introducir el texto mediante voz y dos botones los cuales envían el texto al servidor.

\imagenMediana{CapturaVoz.png}{Reconocimiento de voz}

Dependiendo de la búsqueda que hagamos, podemos encontrar diferentes situaciones de respuesta.

\newpage

\subsubsection{Interacción con el asistente virtual}

Se puede interaccionar con el asistente virtual con mensajes como \textit{hola, buenos días, adiós, etc}.

\imagenMediana{CapturaSaludo.png}{Interacción con el asistente virtual}

\newpage

\subsubsection{Respuesta única}

Tras realizar una consulta con el asistente virtual, se pueden obtener diferentes situaciones de respuesta. Una de ellas, es que la búsqueda tenga una única respuesta. Un ejemplo de esto, puede ser la búsqueda del término \textit{correo}.

\imagenMediana{CapturaUnica.png}{Respuesta única}

\newpage

\subsubsection{Múltiples respuestas}

Otra de las posibles situaciones, es que la búsqueda que hemos realizado tenga múltiples respuestas. En esta situación, el asistente nos enseña todas las posibles respuestas que ha encontrado, para que se elija la que mas se ajusta al objetivo del usuario. Un ejemplo de este tipo de búsqueda es utilizando el término \textit{informática}.

\imagenMediana{CapturaMultiple.png}{Múltiples respuestas}

\newpage

\subsubsection{Sin respuesta}

La última posible situación que se puede llegar a dar, es que la búsqueda que se ha realizado no posee ninguna respuesta. En este caso, el asistente mostrará tres recomendaciones para ver si se ajustan al objetivo de búsqueda realizada por el usuario. Para obtener un ejemplo de este tipo de búsqueda, se puede utilizar el término \textit{Google}.

\imagenMediana{CapturaSin.png}{Pregunta sin respuesta}

\subsubsection{Enviar búsqueda y nueva búsqueda}

En la interfaz, podemos encontrar dos botones, uno con el texto \textit{ENVIAR} y otro con el texto \textit{ŃUEVO}.

El botón \textit{ENVIAR} sirve para enviar la respuesta siguiendo con las búsquedas que se han hecho con anterioridad, de esta manera, el usuario puede ir refinando la búsqueda hasta obtener la respuesta deseada.

\imagenMediana{CapturaEnviar.png}{Continuar búsqueda}

El botón \textit{NUEVO} sirve para reiniciar los términos de la búsqueda, de forma que el usuario puede empezar a realizar una búsqueda desde cero, no teniendo en cuenta el asistente los términos utilizados anteriormente.

\imagenMediana{CapturaNuevo.png}{Reiniciar búsqueda}

\subsubsection{Página web}

Cuando el usuario selecciona una respuesta otorgada por el asistente virtual, se nos abre la página web de la respuesta desde la propia aplicación, pudiendo interactuar con ella sin tener que salir de la aplicación.

\imagenMediana{CapturaWeb.png}{Respuesta consultada}

\subsubsection{Valoración}

Cuando el usuario termina de consultar la página web y vuelve al asistente, una ventana emergente, en la cual se le da la opción de valorar la respuesta obtenida.

\imagenMediana{CapturaValoracion.png}{Valoración de la respuesta}

\subsection{Aplicación web de administración}

Según accedemos a la aplicación de administración, lo que vemos es en la parte superior es el logo de UBUassistant. En la parte superior derecha, nos encontramos con un símbolo de un engranaje y llave inglesa, el cual nos dirige a la página de inicio de sesión de administrador. En la parte inferior izquierda, nos encontramos el logo de la aplicación, el cual si lo pulsamos, se abre el asistente, de tal forma que el administrador puede realizar todas las pruebas pertinentes sobre el funcionamiento del asistente.

\imagen{AdminInterfaz.png}{Interfaz de administración}

Cuando pulsamos sobre el logo situado en la parte derecha superior, se accede a la pantalla en la que el administrador tiene que introducir sus credenciales.

\imagen{AdminLogin.png}{Interfaz de inicio de sesión}

Una vez que el administrador ha iniciado sesión correctamente, automáticamente se abre la interfaz de administración. En la parte superior de la interfaz, se encuentran los botones para que el administrador pueda navegar hacia las diferentes vistas.

Como pantalla principal, aparece el \textit{log} del sistema. En esta ventana, el administrador puede visualizar todas las respuestas que se han abierto con el asistente en forma de tabla. El dato más importante que puede consultar el administrador, es la valoración que el usuario le ha dado a la respuesta.

\imagen{AdminLog.png}{Interfaz de \textit{log}}

En la interfaz de aprendizaje, podemos ver las recomendaciones de los usuarios para que el algoritmo aprenda en los casos de que no puede ofrecer respuestas. Estas mejoras vienen de cuando el algoritmo no encuentra respuesta y de las recomendaciones ofrecidas al usuario, éste abre una de ellas. A parte de ver las recomendaciones, el administrador desde esta página puede aceptar la recomendación, por lo que al asistente lo aprende en el momento; o desechar la recomendación de aprendizaje.

\imagen{AdminAprende.png}{Interfaz de aprendizaje}

En la opción \textit{Editor de casos}, el administrador se encuentra con dos opciones, \textit{Añadir caso} y \textit{Modificar caso}.

\imagenMediana{AdminCasosMenu.png}{Menú editor de casos}

\label{anadirCaso}
En el caso de \textit{Añadir caso}, la interfaz muestra un formulario, con el cual se puede añadir un nuevo caso a la base de datos. El formulario tiene unos campos que son necesarios para poder añadir un caso. Estos campos obligatorios son, \textit{Palabra Clave 1}, \textit{Categoría} y \textit{Respuesta}.

\imagen{AdminCasoAnadir.png}{Interfaz añadir caso}

En la opción \textit{Modificar casos}, nos aparece en forma de tabla todos los casos que posee el asistente virtual. Desde esta interfaz, el administrador puede editar o eliminar un caso.

\imagen{AdminCasoTabla.png}{Interfaz de administración de casos}

En el caso de que se elija la opción \textit{Editar}, se abre una nueva interfaz con un formulario igual al de \textit{Añadir caso} (\ref{anadirCaso}), teniendo que tener rellenados los mismos campos obligatorios. En este caso, el formulario trae campos rellenados según el caso que hayamos elegido para modificar.

\imagen{AdminCasoEditar.png}{Interfaz editar caso}

En el caso de que se elija la opción \textit{Eliminar}, aparece un mensaje de aviso, para preguntarnos si estamos seguros de querer borrar el caso.

\imagen{AdminCasoEliminar.png}{Aviso de borrado de caso}

A parte de todas estas vistas que se han comentado, en el caso de que los datos se muestran en forma de tabla, el administrador tiene la opción de extraer la tabla en diferentes formatos, Esto se puede realizar pulsando el botón \textit{Guardar como}, situado en la parte superior izquierda de las tablas.

\imagen{AdminExport.png}{Exportar tabla}

Otra función importante de las tablas es que son ordenables. De esta forma, el administrador, puede ver los datos de la forma que crea más conveniente. Para ordenar la tabla, solamente hay que pulsar sobre la cabecera de la columna que se quiere ordenar. La primera vez, la tabla se ordena de forma ascendente o de A-Z. La segunda vez, la tabla se ordena de forma descendente o de Z-A.