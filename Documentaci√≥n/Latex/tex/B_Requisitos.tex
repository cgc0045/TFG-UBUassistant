\apendice{Especificación de Requisitos}

\section{Introducción}
En este apartado se va a proceder a explicar de forma detalla como funciona nuestra aplicación y el comportamiento que tiene la misma mientras se usa.

Se va a detallar los diferentes casos de uso que existen para cada una de las actividades que puede realizar un usuario.

Para facilitar la comprensión en este apartado, se va a intentar emplear un lenguaje informal, intentando llegar a cualquier tipo de público.

Los requisitos que nos podemos encontrar son de dos tipos:

\begin{itemize}
	\tightlist
	\item
	\textbf{Requisitos funcionales: } son aquellos requisitos que dictaminan los diferentes servicios que debe de ofrecer un software. Estos requisitos están relacionados con los diferentes casos de uso.
	
	\item
	\textbf{Requisitos no funcionales: } son aquellos que marcan las pautas y las restricciones de diseño y/o implementación del software. 
\end{itemize}


\section{Objetivos generales}

Los objetivos marcados para el desarrollo de este proyecto fueron:

\begin{itemize}
	\tightlist
	\item
	Utilizar la voz para introducir el texto de las búsquedas.
	\item 
	Ampliar el número máximo de vocablos utilizados para realizar las búsquedas.
	\item 
	Disponer de la aplicación en dispositivos móviles con sistema operativo Android.
	\item 
	Analizar otros frameworks de Razonamiento Basado en Casos, como se sugirió en la defensa del proyecto e implementarlo en función del resultado.
	\item 
	Independizar la capa de aplicación de la interfaz de presentación que permita implementar el sistema cliente en distintos dispositivos móviles y distintas tecnologías como HTML5.
	\item 
	Minimizar los posibles ataques de denegación de servicio.
	\item 
	Mantener la imagen corporativa.
\end{itemize}

\section{Catalogo de requisitos}

De lo anteriormente mencionado, obtenemos los siguientes requisitos funcionales:

\begin{itemize}
	\tightlist
	\item
	\textbf{RF-1 Reconocimiento de texto mediante voz: } La aplicación debe de ser capaz de reconocer la voz del usuario y convertirlo a texto.
	\item 
	\textbf{RF-2 Conectar mediante Internet al servidor: } 
\end{itemize}


\section{Especificación de requisitos}


