\apendice{Especificación de Requisitos}

\section{Introducción}
En este apartado se va a proceder a explicar de forma detalla como funciona nuestra aplicación y el comportamiento que tiene la misma mientras se usa.

Se va a detallar los diferentes casos de uso que existen para cada una de las actividades que puede realizar un usuario.

Para facilitar la comprensión en este apartado, se va a intentar emplear un lenguaje informal, intentando llegar a cualquier tipo de público.

Los requisitos que nos podemos encontrar son de dos tipos:

\begin{itemize}
	\tightlist
	\item
	\textbf{Requisitos funcionales: } son aquellos requisitos que dictaminan los diferentes servicios que debe de ofrecer un software. Estos requisitos están relacionados con los diferentes casos de uso.
	
	\item
	\textbf{Requisitos no funcionales: } son aquellos que marcan las pautas y las restricciones de diseño y/o implementación del software. 
\end{itemize}


\section{Objetivos generales}

Los objetivos marcados para el desarrollo de este proyecto fueron:

\begin{itemize}
	\tightlist
	\item
	Utilizar la voz para introducir el texto de las búsquedas.
	\item 
	Ampliar el número máximo de vocablos utilizados para realizar las búsquedas.
	\item 
	Disponer de la aplicación en dispositivos móviles con sistema operativo Android.
	\item 
	Analizar otros frameworks de Razonamiento Basado en Casos, como se sugirió en la defensa del proyecto e implementarlo en función del resultado.
	\item 
	Independizar la capa de aplicación de la interfaz de presentación que permita implementar el sistema cliente en distintos dispositivos móviles y distintas tecnologías como HTML5.
	\item 
	Minimizar los posibles ataques de denegación de servicio.
	\item 
	Mantener la imagen corporativa.
\end{itemize}

\section{Catalogo de requisitos}

El proyecto que se ha desarrollado posee diferentes requisitos de uso, tanto funcionales como no funcionales. Dentro de todos estos requisitos, hay uno que es fundamental en el desarrollo del proyecto. Este requisito, es que la aplicación cliente que se desarrolle, tiene que tener los mismos requisitos que poseía el asistente web realizado por Daniel Santidrián en la versión del proyecto que se ha mejorado. \newline A parte de este requisito, otro muy importante durante el desarrollo de la aplicación, se ha tenido en cuenta que ésta sea compatible con cualquier dispositivo Android con versión igual o superior a 4.4. \newline Con esta premisa, los requisitos funcionales de nuestra aplicación son:

\begin{itemize}
	\tightlist
	\item
	\textbf{RF-1 Reconocimiento de texto mediante voz:} La aplicación debe de ser capaz de reconocer la voz del usuario y convertirlo a texto.
	\item 
	\textbf{RF-2 Reconocimiento de preguntas:} La aplicación tiene que ser capaz de reconocer las preguntas del usuario.
	\item 
	\textbf{RF-3 Obtener respuesta:} El sistema debe de ser capaz de obtener respuestas a la pregunta realizada por el usuario.
	\item 
	\textbf{RF-4 Realizar recomendaciones:} El sistema debe de ser capaz de sugerir respuestas recomendadas cuando el asistente obtiene varias respuesta o no obtiene respuestas.
	\item 
	\textbf{RF-5 Navegador web:} La aplicación cliente tiene que ser capaz de abrir los enlaces ofrecidos por el asistente sin recurrir a un navegador web externo.
	\item 
	\textbf{RF-6 Valoración de respuestas:} La aplicación tiene que ser capaz de mostrarle al usuario la posibilidad de valorar la respuesta dada por parte del asistente.
	\item 
	\textbf{RF-7 Reiniciar palabras de búsqueda:} La aplicación debe de permitir al usuario que reinicie las palabras con las que se efectúan la búsqueda, permitiendo así, empezar una nueva búsqueda desde cero.
	\item 
	\textbf{RF-8 Gestión del log:} El administrador de la aplicación debe de ser capaz de gestionar el registro de las búsquedas realizadas por el usuario.
	\begin{itemize}
		\item 
		\textbf{RF-8.1 Listar log:} La aplicación debe ser capaz de listar el registro de las búsquedas realizadas.
	\end{itemize}
	\item 
	\textbf{RF-9 Gestión de casos:} La aplicación tiene que permitir gestionar a través de una interfaz los diferentes casos que se encuentran almacenados.
	\begin{itemize}
		\item 
		\textbf{RF-9.1 Listar casos:} La interfaz tiene que ser capaz de listar todos los casos almacenados correctamente.
		\item 
		\textbf{RF-9.2 Añadir caso:} El sistema debe permitir añadir nuevos casos a la base de datos
		\item 
		\textbf{RF-9.3 Editar caso:} La aplicación tiene que permitir al administrador modificar los casos almacenados en la base de datos.
	\end{itemize}
	\item 
	\textbf{RF-10 Gestión de aprendizaje:} El sistema tiene que permitir que el administrador pueda gestionar el aprendizaje del algoritmo.
	\begin{itemize}
		\item 
		\textbf{RF-10.1 Aprender recomendación:} La aplicación tiene que ser capaz de aprender las diferentes recomendaciones que hayan sido aprobadas por el administrador.
	\end{itemize}
\end{itemize}

A parte de los anteriores requisitos funcionales, tenemos requisitos no funcionales:

\begin{itemize}
	\tightlist
	\item 
	\textbf{RNF-1 Seguridad:} El servidor tiene que estar protegido ante los posibles ataques de denegación de servicio.
	\item 
	\textbf{RNF-2 Mejorar búsqueda:} Ampliar el número de vocablos de 5 a 7, para facilitar así al usuario las búsquedas.
	\item 
	\textbf{RNF-3 Adaptabilidad:} La aplicación tiene que ser adaptable fácilmente a diferentes plataformas.
	\item 
	\textbf{RNF-4 Imagen corporativa:} Mantener la imagen corporativa de la Universidad de Burgos.
	\item 
	\textbf{RNF-5 Registro de actividad:} La aplicación debe de ser capaz de registrar los diferentes enlaces accedidos por un usuario.
	\item 
	\textbf{RNF-6 Fácil manejo:} La aplicación tiene que ser sencilla de utilizar, tanto como para poder enviar las preguntas, como de visualizar las respuestas ofrecidas por el asistente. 
	\item 
	\textbf{RNF-7 Servicio virtualizado:} El servidor de la aplicación tiene que estar virtualizado en la nube, como \textit{Microsoft Azure}, permitiendo así que sea accesible desde cualquier localización.
\end{itemize}

\newpage

\section{Especificación de requisitos}

\subsection{Diagrama de casos de uso}

\imagen{DiagramaCasos.png}{Diagrama de casos de uso}

\subsection{Actores}

En el sistema que se ha desarrollado, nos podemos encontrar con dos perfiles actores totalmente diferenciados:

\begin{itemize}
	\item 
	\textbf{Usuario:} El usuario es el que se encarga de utilizar la aplicación Android para realizar las diferentes búsquedas que considere pertinentes.
	\item 
	\textbf{Administrador:} El administrador utiliza la interfaz web con sus credenciales, para gestionar los diferentes ámbitos del asistente de búsqueda.
\end{itemize}


\newpage

\subsection{Casos de uso}


\tablaSinColores{CU-01 Preguntar}
{p{4cm} p{10cm}}{2}{CU-01}
{\textbf{CU-01} & \textbf{Preguntar}\\}{
	\textbf{Versión} 				& 1.0\\
	\textbf{Autor} 					& Carlos González Calatrava\\
	\textbf{Requisitos asociados} 	& RF-1 \& RF-2 \\
	\textbf{Descripción} 			& Permite al usuario introducir texto en la interfaz del asistente para realizar una pregunta. \\
	\textbf{Precondiciones} 		& Se ha conectado con la base de datos. \\
	\textbf{Acciones}				& El usuario abre la aplicación Android. \\
	 								& Introduce texto o lo dicta por voz. \\
	 								& Enviar la pregunta. \\
	\textbf{Postcondiciones}		& Obtener resultados de la respuesta. \\
	\textbf{Excepciones}			& Servidor web no disponible. \\
}

\tablaSinColores{CU-02 Obtener respuesta}
{p{4cm} p{10cm}}{2}{CU-02}
{\textbf{CU-02} & \textbf{Obtener respuesta}\\}{
	\textbf{Versión} 				& 1.0\\
	\textbf{Autor} 					& Carlos González Calatrava\\
	\textbf{Requisitos asociados} 	& RF-3\\
	\textbf{Descripción} 			& El usuario obtiene una respuesta a la pregunta realizada. \\
	\textbf{Precondiciones} 		& Se ha conectado con la base de datos. \\
									& Existe respuesta única para la pregunta realizada. \\
	\textbf{Acciones}				& El usuario abre la aplicación Android. \\
									& Introduce texto o lo dicta por voz. \\
									& Enviar la pregunta. \\
									& Obtener la respuesta a la pregunta. \\
	\textbf{Postcondiciones}		& Se almacena la respuesta en el log del servidor \\
	\textbf{Excepciones}			& Servidor web no disponible. \\
}

\newpage

\tablaSinColores{CU-03 Obtener recomendación}
{p{4cm} p{10cm}}{2}{CU-03}
{\textbf{CU-03} & \textbf{Obtener recomendación}\\}{
	\textbf{Versión} 				& 1.0\\
	\textbf{Autor} 					& Carlos González Calatrava\\
	\textbf{Requisitos asociados} 	& RF-3 \& RF-4\\
	\textbf{Descripción} 			& El usuario obtiene múltiples posibles respuestas a la pregunta realizada o recomendaciones por que la pregunta no tiene respuesta. \\
	\textbf{Precondiciones} 		& Se ha conectado con la base de datos. \\
									& Existe múltiples respuestas o no existen respuestas para la pregunta realizada. \\
	\textbf{Acciones}				& El usuario abre la aplicación Android. \\
									& Introduce texto o lo dicta por voz. \\
									& Enviar la pregunta. \\
									& El servidor no obtiene respuesta o encuentra múltiples respuestas. \\
									& Se muestran las recomendaciones a la pregunta. \\
	\textbf{Postcondiciones}		& Se almacena la respuesta en el log del servidor \\
	\textbf{Excepciones}			& Servidor web no disponible. \\
}

\tablaSinColores{CU-04 Visualizar respuesta}
{p{4cm} p{10cm}}{2}{CU-04}
{\textbf{CU-04} & \textbf{Visualizar respuesta}\\}{
	\textbf{Versión} 				& 1.0\\
	\textbf{Autor} 					& Carlos González Calatrava\\
	\textbf{Requisitos asociados} 	& RF-5\\
	\textbf{Descripción} 			& El usuario puede visualizar la página web dada como respuesta desde la propia aplicación. \\
	\textbf{Precondiciones} 		& Se ha conectado con la base de datos. \\
									& Existe respuesta única para la pregunta realizada. \\
	\textbf{Acciones}				& El usuario abre la aplicación Android. \\
									& Introduce texto o lo dicta por voz. \\
									& Enviar la pregunta. \\
									& Obtener la respuesta a la pregunta. \\
									& El usuario selecciona la respuesta. \\
									& La respuesta se abre dentro de la propia aplicación. \\
	\textbf{Postcondiciones}		& Se valora la respuesta recibida. \\
	\textbf{Excepciones}			& Servidor web no disponible. \\
}

\tablaSinColores{CU-05 Visualizar recomendación}
{p{4cm} p{10cm}}{2}{CU-05}
{\textbf{CU-05} & \textbf{Visualizar recomendación}\\}{
	\textbf{Versión} 				& 1.0\\
	\textbf{Autor} 					& Carlos González Calatrava\\
	\textbf{Requisitos asociados} 	& RF-5\\
	\textbf{Descripción} 			& El usuario puede visualizar la página web respuesta que seleccione entre las diferentes recomendaciones realizadas por el asistente. \\
	\textbf{Precondiciones} 		& Se ha conectado con la base de datos. \\
									& Existe múltiples respuestas o no existen respuestas para la pregunta realizada. \\
	\textbf{Acciones}				& El usuario abre la aplicación Android. \\
									& Introduce texto o lo dicta por voz. \\
									& Enviar la pregunta. \\
									& El servidor no obtiene respuesta o encuentra múltiples respuestas. \\
									& Se muestran las recomendaciones a la pregunta. \\
									& El usuario selecciona una de las recomendaciones realizadas. \\
									& La recomendación seleccionada se abre dentro de la propia aplicación. \\		
	\textbf{Postcondiciones}		& Se valora la recomendación seleccionada. \\
	\textbf{Excepciones}			& Servidor web no disponible. \\
}

\newpage

\tablaSinColores{CU-06 Valorar respuesta}
{p{4cm} p{10cm}}{2}{CU-06}
{\textbf{CU-06} & \textbf{Valorar respuesta}\\}{
	\textbf{Versión} 				& 1.0\\
	\textbf{Autor} 					& Carlos González Calatrava\\
	\textbf{Requisitos asociados} 	& RF-6\\
	\textbf{Descripción} 			& El usuario puede valorar la respuesta o la recomendación que ha seleccionado. \\
	\textbf{Precondiciones} 		& Se ha conectado con la base de datos. \\
									& Existe múltiples respuestas o no existen respuestas para la pregunta realizada. \\
									& La página web de la respuesta se encuentra disponible. \\
	\textbf{Acciones}				& El usuario abre la aplicación Android. \\
									& Introduce texto o lo dicta por voz. \\
									& Enviar la pregunta. \\
									& El servidor no obtiene respuesta o encuentra múltiples respuestas (recomendación) o tiene respuesta única. \\
									& Se muestran las recomendaciones o la respuesta a la pregunta. \\
									& El usuario selecciona una de las recomendaciones realizadas o la respuesta. \\
									& La recomendación seleccionada o respuesta se abre dentro de la propia aplicación. \\
									& El usuario vuelve al asistente. \\
									& Valora la respuesta que se le ha otorgado. \\		
	\textbf{Postcondiciones}		& Se almacena la valoración en la base de datos. \\
	\textbf{Excepciones}			& Servidor web no disponible. \\
}

\tablaSinColores{CU-07 Reiniciar búsqueda}
{p{4cm} p{10cm}}{2}{CU-07}
{\textbf{CU-07} & \textbf{Reiniciar búsqueda}\\}{
	\textbf{Versión} 				& 1.0\\
	\textbf{Autor} 					& Carlos González Calatrava\\
	\textbf{Requisitos asociados} 	& RF-7\\
	\textbf{Descripción} 			& El usuario debe de ser capaz de reiniciar las palabras que se están utilizando para hacer la búsqueda. \\
	\textbf{Precondiciones} 		& Se han realizado búsquedas con anterioridad. \\
	\textbf{Acciones}				& El usuario abre la aplicación Android. \\
									& Introduce texto o lo dicta por voz. \\
									& Pulsa sobre el botón con el texto \textit{Nueva}.	\\
	\textbf{Postcondiciones}		& Se realiza la búsqueda con las nuevas palabras. \\
	\textbf{Excepciones}			& No se han realizado búsquedas con anterioridad. \\
}

\tablaSinColores{CU-08 Gestión del log}
{p{4cm} p{10cm}}{2}{CU-08}
{\textbf{CU-08} & \textbf{Gestión del log}\\}{
	\textbf{Versión} 				& 1.0\\
	\textbf{Autor} 					& Carlos González Calatrava\\
	\textbf{Requisitos asociados} 	& RF-8 \& RF-8.1 \\
	\textbf{Descripción} 			& Permite al administrador gestionar el registro de las búsquedas realizadas con el asistente virtual. \\
	\textbf{Precondiciones} 		& La base de datos se encuentra accesible. \\
									& El administrador se ha autenticado en el sistema con sus credenciales. \\
	\textbf{Acciones}				& El usuario accede a la página web. \\
									& El usuario abre la interfaz de administración. \\
									& El usuario se autentica en el sistema con sus credenciales de administrador.	\\
									& El administrador accede a la página de gestión del log. \\
									& Se listan las diferentes búsquedas realizadas con el asistente virtual. \\
									& Se permite ordenar las búsquedas por diferentes campos. \\
									& Se permite exportar la lista a diferentes formatos. \\
									& Se permite borrar la lista de búsquedas realizadas. \\
	\textbf{Postcondiciones}		& El número de logs que se muestran es igual al número de logs que se encuentra en la base de datos. \\
	\textbf{Excepciones}			& La base de datos no se encuentra accesible. \\
}

\tablaSinColores{CU-09 Listar log}
{p{4cm} p{10cm}}{2}{CU-09}
{\textbf{CU-09} & \textbf{Listar log}\\}{
	\textbf{Versión} 				& 1.0\\
	\textbf{Autor} 					& Carlos González Calatrava\\
	\textbf{Requisitos asociados} 	& RF-8.1 \\
	\textbf{Descripción} 			& Permite al administrador visualizar en forma de tabla el registro de todas las búsquedas que ha realizado un usuario utilizando el asistente virtual. \\
	\textbf{Precondiciones} 		& La base de datos se encuentra accesible. \\
									& El administrador se ha autenticado en el sistema con sus credenciales. \\
	\textbf{Acciones}				& El usuario accede a la página web. \\
									& El usuario abre la interfaz de administración. \\
									& El usuario se autentica en el sistema con sus credenciales de administrador.	\\
									& El administrador accede a la página de gestión del log. \\
									& Se listan las diferentes búsquedas realizadas con el asistente virtual. \\
	\textbf{Postcondiciones}		& El número de logs que se muestran es igual al número de logs que se encuentra en la base de datos. \\
	\textbf{Excepciones}			& La base de datos no se encuentra accesible. \\
}

\tablaSinColores{CU-10 Gestión de casos}
{p{4cm} p{10cm}}{2}{CU-10}
{\textbf{CU-10} & \textbf{Gestión de casos}\\}{
	\textbf{Versión} 				& 1.0\\
	\textbf{Autor} 					& Carlos González Calatrava\\
	\textbf{Requisitos asociados} 	& RF-9, RF-9.1, RF-9.2 \& RF-9.3 \\
	\textbf{Descripción} 			& Permite al administrador gestionar los diferentes casos que son utilizados por el asistente virtual. \\
	\textbf{Precondiciones} 		& La base de datos se encuentra accesible. \\
									& El administrador se ha autenticado en el sistema con sus credenciales. \\
	\textbf{Acciones}				& El usuario accede a la página web. \\
									& El usuario abre la interfaz de administración. \\
									& El usuario se autentica en el sistema con sus credenciales de administrador.	\\
									& El administrador accede a las páginas de las sección \textit{Editor de casos }  \\
									& Si se accede a la página \textit{Añadir Caso}, se muestra un formulario, con el cual se puede añadir un nuevo caso a la base de datos para que lo pueda utilizar el asistente virtual.\\
									& Si se accede a la página \textit{Modificar Casos}, se muestra en forma de tabla los diferentes casos que utiliza el asistente virtual. \\
									& Se muestran diferentes menús con los que se permite editar o eliminar cada uno de los diferentes casos. \\
									& Se muestra una opción que permite exportar la tabla a diferentes formatos. \\
	\textbf{Postcondiciones}		& El número de casos que se muestran es igual al número de casos que se encuentra en la base de datos. \\
	\textbf{Excepciones}			& La base de datos no se encuentra accesible. \\
}

\tablaSinColores{CU-11 Listar casos}
{p{4cm} p{10cm}}{2}{CU-11}
{\textbf{CU-11} & \textbf{Listar casos}\\}{
	\textbf{Versión} 				& 1.0\\
	\textbf{Autor} 					& Carlos González Calatrava\\
	\textbf{Requisitos asociados} 	& RF-9.1 \\
	\textbf{Descripción} 			& Permite al administrador ver en forma de tabla los casos almacenados en la base de datos. \\
	\textbf{Precondiciones} 		& La base de datos se encuentra accesible. \\
									& El administrador se ha autenticado en el sistema con sus credenciales. \\
	\textbf{Acciones}				& El usuario accede a la página web. \\
									& El usuario abre la interfaz de administración. \\
									& El usuario se autentica en el sistema con sus credenciales de administrador.	\\
									& El administrador accede a la página \textit{Modificar Casos}. \\
									& Se muestran todos los casos almacenados en forma de tabla. \\
	\textbf{Postcondiciones}		& El número de casos que se muestran es igual al número de casos que se encuentra en la base de datos. \\
	\textbf{Excepciones}			& La base de datos no se encuentra accesible. \\
}

\tablaSinColores{CU-12 Añadir caso}
{p{4cm} p{10cm}}{2}{CU-12}
{\textbf{CU-12} & \textbf{Añadir caso}\\}{
	\textbf{Versión} 				& 1.0\\
	\textbf{Autor} 					& Carlos González Calatrava\\
	\textbf{Requisitos asociados} 	& RF-9.2 \\
	\textbf{Descripción} 			& Permite al administrador añadir un nuevo caso a la base de datos para que lo pueda utilizar el asistente virtual. \\
	\textbf{Precondiciones} 		& La base de datos se encuentra accesible. \\
									& El administrador se ha autenticado en el sistema con sus credenciales. \\
	\textbf{Acciones}				& El usuario accede a la página web. \\
									& El usuario abre la interfaz de administración. \\
									& El usuario se autentica en el sistema con sus credenciales de administrador.	\\
									& El administrador accede a la página \textit{Añadir Caso}. \\
									& Se muestra el formulario para poder añadir un nuevo caso. \\
									& El administrador rellena los campos obligatorios y aquellos opcionales que considere oportunos. \\
									& Pulsa sobre el botón de \textit{Aceptar} para añadir el nuevo caso. \\
	\textbf{Postcondiciones}		& El caso que se ha añadido se encuentra en la base de datos. \\
	\textbf{Excepciones}			& La base de datos no se encuentra accesible. \\
									& Los campos marcados como obligatorios no se han rellenado. \\
}

\tablaSinColores{CU-13 Editar caso}
{p{4cm} p{10cm}}{2}{CU-13}
{\textbf{CU-13} & \textbf{Editar caso}\\}{
	\textbf{Versión} 				& 1.0\\
	\textbf{Autor} 					& Carlos González Calatrava\\
	\textbf{Requisitos asociados} 	& RF-9.3 \\
	\textbf{Descripción} 			& Permite al administrador editar un caso que ya existe en la base de datos. \\
	\textbf{Precondiciones} 		& La base de datos se encuentra accesible. \\
									& El administrador se ha autenticado en el sistema con sus credenciales. \\
									& El caso que se quiere modificar, existe. \\
	\textbf{Acciones}				& El usuario accede a la página web. \\
									& El usuario abre la interfaz de administración. \\
									& El usuario se autentica en el sistema con sus credenciales de administrador.	\\
									& El administrador accede a la página \textit{Modificar Caso}. \\
									& Se muestra la tabla con todos los casos almacenados en la base de datos. \\
									& El administrador pulsa sobre el botón \textit{Editar} del caso que quiere modificar. \\
									& Se muestra un formulario con los datos del caso a modificar. \\
									& El administrador cambia o rellena los campos que considere oportuno, dejando siempre rellenados los campos que son obligatorios. \\
									& El administrador pulsa sobre el botón \textit{Aceptar} para modificar el caso. \\
	\textbf{Postcondiciones}		& El caso que se ha añadido se encuentra en la base de datos. \\
	\textbf{Excepciones}			& La base de datos no se encuentra accesible. \\
									& Los campos marcados como obligatorios no se han rellenado. \\
}

\newpage

\tablaSinColores{CU-14 Gestión de aprendizaje}
{p{4cm} p{10cm}}{2}{CU-14}
{\textbf{CU-14} & \textbf{Gestión de aprendizaje}\\}{
	\textbf{Versión} 				& 1.0\\
	\textbf{Autor} 					& Carlos González Calatrava\\
	\textbf{Requisitos asociados} 	& RF-10 \& RF-10.1 \\
	\textbf{Descripción} 			& Permite al administrador gestionar los diferentes casos que son utilizados por el asistente virtual. \\
	\textbf{Precondiciones} 		& La base de datos se encuentra accesible. \\
									& El administrador se ha autenticado en el sistema con sus credenciales. \\
	\textbf{Acciones}				& El usuario accede a la página web. \\
									& El usuario abre la interfaz de administración. \\
									& El usuario se autentica en el sistema con sus credenciales de administrador.	\\
									& El administrador accede a la página \textit{Aprendizaje}. \\
									& Se muestra una tabla con todas las recomendaciones que ha ido recogiendo el asistente virtual. \\
									& Se muestran opciones para aprende o descartar la recomendación de aprendizaje. \\
									& Se muestra una opción para poder exportar la tabla de recomendaciones a diferentes formatos. \\
	\textbf{Postcondiciones}		& El número de recomendaciones que se muestran es igual al número de recomendaciones que se encuentra en la base de datos. \\
	\textbf{Excepciones}			& La base de datos no se encuentra accesible. \\
}

\tablaSinColores{CU-15 Aprender recomendación}
{p{4cm} p{10cm}}{2}{CU-15}
{\textbf{CU-15} & \textbf{Aprender recomendación}\\}{
	\textbf{Versión} 				& 1.0\\
	\textbf{Autor} 					& Carlos González Calatrava\\
	\textbf{Requisitos asociados} 	& RF-10.1 \\
	\textbf{Descripción} 			& Permite al administrador considerar como buena la recomendación de aprendizaje del asistente virtual. \\
	\textbf{Precondiciones} 		& La base de datos se encuentra accesible. \\
									& El administrador se ha autenticado en el sistema con sus credenciales. \\
									& Existe al menos una recomendación de aprendizaje. \\
	\textbf{Acciones}				& El usuario accede a la página web. \\
									& El usuario abre la interfaz de administración. \\
									& El usuario se autentica en el sistema con sus credenciales de administrador.	\\
									& El administrador accede a la página \textit{Aprendizaje}. \\
									& Se muestra una tabla con todas las recomendaciones que ha ido recogiendo el asistente virtual. \\
									& El administrador pulsa sobre la opción \textit{Aprender} de la recomendación que desee. \\
	\textbf{Postcondiciones}		& El número de recomendaciones que se muestran es igual al número de recomendaciones que se encuentra en la base de datos. \\
									& El nuevo caso se encuentra disponible en la base de datos. \\
									& La recomendación ha sido eliminada de la interfaz de aprendizaje. \\
	\textbf{Excepciones}			& La base de datos no se encuentra accesible. \\
}


