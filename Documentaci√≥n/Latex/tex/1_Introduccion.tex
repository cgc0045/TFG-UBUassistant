\capitulo{1}{Introducción}

Para ciertas personas, poder encontrar información desde un \textit{smartphone} les supone una tarea bastante tediosa y complicada, lo cuál les lleva muchas veces a la deseperación y finalmente no acaban encontrando aquello que deseaban encontrar.

La página web de la Universidad de Burgos es una de las mejores adaptadas a los diferentes dispositivos con los que se puede acceder a la misma. Pero como la gran mayoría de páginas web de universidades, esta contiene mucha información, como es lógico, haciendo que para ciertas personas pueda ser algo complicado navegar a través de ella. Esto se debe que a pesar de que la página esté estructurada en cinco menús, la cantidad de submenús que se encuentran dentro de ellos pueden llegar a confundir al usuario cuando a accedido a varios de ellos desde estos dispositivos, haciéndole dudar como ha conseguido entrar hasta el mismo para futuras consultas.

El método propuesto para facilitar la búsqueda de estos usuarios es una aplicación para dispositivos Android, ya que en España el 87,1 \% de los dispositivos son Android \cite{android:data}. De esta forma se pretende facilitar el uso de la web de la Universidad de Burgos al mayor porcentaje posible de personas.

El asistente es el encargado de buscar una respuesta al texto introducido por el usuario dentro de la aplicación. El usuario no tiene que introducir palabras clave para que el asistente funcione correctamente, sino que puede emplear lenguaje natural, ya que los algoritmos del asistente es el encargado de analizar el texto introducido y poder encontrar así la respuesta más adecuada para el usuario.

\section{Estructura de la memoria}\label{estructura-de-la-memoria}
La memoria tiene la siguiente división en apartados:

\begin{itemize}
	\tightlist
	\item
	\textbf{Introducción:} en este apartado se realiza una descripción de una manera breve del problema que se intenta resolver y la solución otorgada. Además incluye subapartados con la estructura de la memoria y el listado de materiales adjuntos.
	\item
	\textbf{Objetivos del proyecto:} sección donde se explican los objetivos de desarrollar un proyecto de estas características.
	\item
	\textbf{Conceptos teóricos:} capítulo en el que se abordan los conceptos teóricos necesarios para comprender el resultado final del proyecto.
	\item
	\textbf{Técnicas y herramientas:} en esta sección se describen las herramientas y las técnicas que se han utilizado para el desarrollo y gestión del proceso del proyecto.
	\item
	\textbf{Aspectos relevantes del desarrollo:} apartado donde se tratan aquellos aspectos que se consideran destacados en el desarrollo del proyecto.
	\item
	\textbf{Trabajos relacionados:} capitulo que expone y describe aquellos trabajos que están relacionados con la temática de asistente virtual.
	\item
	\textbf{Conclusiones y líneas de trabajo futuras:} sección que explica las conclusiones obtenidas tras la realización del proyecto y la funcionalidad que es posible añadir en el futuro.
\end{itemize}

Además, se proporcionan los siguientes anexos:

\begin{itemize}
	\tightlist
	\item
	\textbf{Plan del proyecto software:} capítulo donde se expone planificación temporal del proyecto y su viabilidad.
	\item
	\textbf{Especificación de requisitos:} en este apartado se desarrollan los objetivos del software y la especificación de requisitos.
	\item
	\textbf{Especificación de diseño:} sección que describe el diseño de datos, el diseño procedimental y el diseño arquitectónico.
	\item
	\textbf{Documentación técnica de programación:} en este capítulo se explica todo lo relacionado con la programación, la estructura de directorios, el manual del programador y las pruebas realizadas.
	\item
	\textbf{Documentación de usuario:} apartado que realiza un explicación sobre los requisitos de usuarios, la instalación y proporciona un manual de usuario.
\end{itemize}

\section{Materiales adjuntos}\label{materiales-adjuntos}

Los materiales que se adjuntan con la memoria son: 

\begin{itemize}
	\tightlist
	\item
	Aplicación Java servlet UBUassistant.
	\item
	Aplicación Android cliente UBUassistant.
	\item
	Máquina virtual Ubuntu configurada de la misma manera que está configurado en Microsoft Azure.
	\item	
	JavaDoc.
\end{itemize}

Además, los siguientes recursos están accesibles a través de internet:

\begin{itemize}
	\tightlist
	\item
	Repositorio del proyecto. \cite{ubuassistant:repo}
\end{itemize}
