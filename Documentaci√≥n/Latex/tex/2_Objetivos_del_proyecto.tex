\capitulo{2}{Objetivos del proyecto}

En este apartado se explica cuáles han sido los objetivos que se han perseguido durante el proyecto.

\section{Objetivos software}\label{objetivos-software}

\begin{itemize}
	\tightlist
	\item
	A nivel de usuario de la aplicación:
	\begin{itemize}
		\tightlist
		\item
		Disponer de la aplicación en dispositivos móviles que utilicen sistema operativo Android.
		\item
		Dentro de la aplicación Android, poder utilizar la voz como herramienta para poder introducir el texto de las búsquedas.
		\item 
		Ampliar el número máximo de vocablos que se utilizan para encontrar los resultados dentro del algoritmo CBR.
	\end{itemize}
	\item 
	A nivel de gestor y desde el punto de vista tecnológico:
	\begin{itemize}
		\tightlist
		\item
		Analizar diferentes frameworks de Razonamiento Basado en casos, tal y como se sugirió en la defensa del TFG anterior. En función del resultado del análisis, implementarlo o no.
		\item 
		Independizar la capa de presentación de la capa de aplicación de tal forma que se permita implementar el sistema cliente a diferentes dispositivos y tecnologías como HTML5, Android o iOS.
		\item 
		Minimizar los posibles ataques de denegación de servicio.
		\item 
		Mantener la imagen corporativa.
	\end{itemize}
\end{itemize}

\section{Objetivos técnicos}\label{objetivos-tecnicos}

\begin{itemize}
	\tightlist
	\item
	Aplicar Scrum, en la medida de lo posible, como metodología de desarrollo ágil.
	\item
	Realizar la aplicación siguiendo el concepto de Modelo Vista Controlador, separando la interfaz de usuario, el motor de la aplicación y los datos.
	\item
	Obtener las respuestas del motor de la aplicación empleando un estándar, haciendo así que la aplicación sea más modular y multiplataforma.
	\item
	Acceder a una base de datos MySQL mediante Hibernate y JDBC.
	\item
	Servirse de GitHub como sistema de control de versiones.
	\item
	Adquirir conocimientos para el desarrollo de aplicaciones en plataforma Android.
	\item 
	Utilizar una máquina virtual de Microsoft Azure para alojar el servidor de la aplicación, para que el asistente sea accesible con cualquier tipo de conexión a Internet.
	
\end{itemize}

\section{Objetivos personales}\label{objetivos-personales}

\begin{itemize}
	\tightlist
	\item
	Emplear la mayor cantidad posible de conocimientos adquiridos durante la carrera.
	\item
	Ampliar los conocimientos adquiridos durante la carrera utilizando nuevos lenguajes de programación (Android).
	\item 
	Conseguir una aplicación que facilite las tareas de búsqueda en la página de la Universidad de Burgos.
\end{itemize}
