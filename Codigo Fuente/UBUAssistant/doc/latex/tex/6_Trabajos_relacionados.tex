\capitulo{6}{Trabajos relacionados}

Existen diferentes métodos de implementar un asistente virtual que permita el reconocimiento de texto. La amplia mayoría utilizan algoritmos de inteligencia artificial, siendo algunos de ellos los que se decantan por el razonamiento basado en casos.

\section{Artículos de investigación}

\subsection{Comportamiento Adaptable de Chatbots Dependiente del Contexto}

Artículo desarrollado por el grupo de investigación en sistemas de
Información (GISI) de la Universidad Nacional de Lanús en Argentina, en el que se estudian los diferentes algoritmos utilizados en asistentes virtuales o \emph{chatbots} centrándose en las mejoras que pueden introducirse en el reconocimiento del contexto. En el artículo se plantea el razonamiento basado en casos como una de las soluciones \cite{rev:invest}.

\subsection{Un Sofbot de Charla Desarrollado con Técnicas de Razonamiento basado en Casos}

En este artículo de la Revista de Investigación de Sistemas e Informática de la Universidad Nacional Mayor de San Marcos se exponen las técnicas utilizadas para la construcción de un software de charla. El software construido utiliza la técnica de razonamiento basado en casos y se explican los distintos pasos realizados para su creación \cite{SAMI:project}.


\section{Proyectos}

\subsection{ChatBot}

Es un proyecto \emph{Open Source} que desarrolla una aplicación web donde se hospeda un chat construido mediante razonamiento basado en casos. Utiliza el framework de jCOLIBRI para llevarlo a cabo. Es un pequeño proyecto desarrollado por un estudiante de la Universidad de Auckland \cite{ChatBot:project}.

\begin{itemize}
\tightlist
\item
Web del proyecto: https://github.com/agkphysics/ChatBot
\end{itemize}

\subsection{SAMI}

En el anterior artículo de investigación \cite{SAMI:project}, se explican las técnicas seguidas para la realización de un software de charla basado en razonamiento basado en casos. El finalidad de este software consiste en proporcionar a los alumnos del aula virtual de la universidad información acerca de la misma. El software reconoce el diálogo escrito.

Este proyecto no dispone de página web y no ha sido encontrada más información a parte de la documentación.

\section{Fortalezas y debilidades del proyecto}

Las fortalezas de este proyecto radican en:

\begin{itemize}
\tightlist
\item
La interfaz de usuario es muy sencilla.
\item
La interfaz es completa y se permite interactuar al usuario con ella.
\item
La aplicación es adaptable a cualquier ámbito solo con cambiar los casos de la base de datos.
\item
En cualquier momento es posible cambiar o retocar los casos, tanto descriptivos como respuesta.
\item
La aplicación dispone de una página de administración para consultar el log de uso y sugerencias de aprendizaje así como la inclusión, la modificación y el borrado de casos.
\item
El asistente virtual puede incluirse en cualquier página web.
\item
La interfaz de la aplicación es adaptable a distintos tamaños de pantalla.
\end{itemize}

Por contra, existen ciertas debilidades como:

\begin{itemize}
\tightlist
\item
La aplicación no está hospedada en ningún sitio web, sólo es accesible mediante una conexión local quedando como una propuesta de futuro poder integrarse en algún servicio de la Universidad de Burgos.
\item
Actualmente solo está preparada para trabajar con casos que tratan sobre la Universidad de Burgos.
\item
No es todo lo precisa que un usuario podría desear, cubre los apartados más amplios de la Universidad.
\end{itemize}

