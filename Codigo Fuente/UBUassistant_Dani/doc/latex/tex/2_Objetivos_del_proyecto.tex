\capitulo{2}{Objetivos del proyecto}

En este apartado se detallan de forma precisa los distintos tipos de objetivos que se han perseguido con la realización del proyecto.

\section{Objetivos software}\label{objetivos-software}

\begin{itemize}
\tightlist
\item
  Desarrollar una aplicación web que permita a los usuarios de la página web de la Universidad de Burgos disponer de un asistente virtual con el que poder interactuar en busca de respuestas y apartados concretos del sitio web.
\item
  Favorecer la búsqueda de información de forma simple y natural.
\item
  Realizar un aprendizaje de nuevos casos en los que no existe respuesta de manera supervisada.
\item
   Dotar a un administrador del log de uso del asistente virtual en busca de perfeccionar respuestas a conceptos más buscados, así como ofrecer la posibilidad de añadir nuevos casos, y editar o eliminar los existentes.
\end{itemize}

\section{Objetivos técnicos}\label{objetivos-tecnicos}

\begin{itemize}
\tightlist
\item
Implementar un algoritmo para buscar respuestas a un texto introducido expresado de forma natural a través del análisis de palabras clave y un framework de razonamiento basado en casos.
\item
Llevar a cabo una página web compatible con todos los navegadores web con el uso de HTML5 y CSS3.
\item
Realizar test unitarios, de integración y de interfaz con distintas herramientas como JUnit y Selenium.
\item
Aplicar Scrum, en la medida de lo posible, como metodología de desarrollo ágil.
\item
Realizar la aplicación siguiendo el concepto de Modelo Vista Controlador, separando la interfaz, el motor de la aplicación y los datos.
\item
Acceder a una base de datos MySQL mediante Hibernate y JDBC.
\item
Servirse de GitHub como sistema de control de versiones.
\item
Utilizar herramientas de control de calidad del software como SonarQube, RefactorIt o InCode.

\end{itemize}

\section{Objetivos personales}\label{objetivos-personales}

\begin{itemize}
\tightlist
\item
  Llevar a cabo una aplicación que en un futuro pueda aportar un valor añadido a la página web de la Universidad de Burgos.
\item
Intentar solucionar un problema común a la hora de buscar información sobre una página web con mucho contenido como es la de la Universidad de Burgos.
\item
Mejorar los conceptos adquiridos en la programación orientada a objetos y programación web distribuida.
\item
Utilizar el mayor número de conocimientos adquiridos durante la carrera.
\end{itemize}

