\apendice{Documentación de usuario}

\section{Introducción}

La documentación de usuario recoge y detalla los requisitos de usuario para ejecutar la aplicación, las instrucciones para su instalación, y las directrices para su utilización.

\section{Requisitos de usuarios}

Para poder utilizar la aplicación el usuario debe disponer de:

\begin{itemize}
\tightlist
\item Dispositivo con navegador web.
\item Navegador web con JavaScript habilitado.
\item Acceso a Internet.
\end{itemize}

\section{Instalación}

La aplicación no requiere instalación. Actualmente se ejecuta en un servidor local, pudiendo acceder a la página web mediante la dirección local de la máquina junto con el puerto que utiliza.

En un futuro el asistente virtual puede ser integrado en cualquier página web por lo que tampoco es necesaria una instalación para poder utilizarla, basta únicamente con acceder a la dirección web donde esté alojada.


\section{Manual del usuario}

EL siguiente manual tratará de orientar y enseñar a un posible usuario del asistente virtual así como a un administrador de la aplicación.

\subsection{Asistente virtual}

\subsubsection{Preguntar}

\begin{enumerate}
\item Acceder a la página donde se encuentra el asistente.
\item Abrir el asistente pulsando sobre el icono.
\imagenPeque{Icon.png}{Icono que abre el asistente.}
\item Realizar la pregunta sobre el apartado "Pregunta a UBUassistant" de la interfaz cargada.
\imagen{Interfaz1.png}{Interfaz asistente virtual.}
\end{enumerate}

\subsubsection{Obtener respuesta}

\begin{enumerate}
\item Acceder a la página donde se encuentra el asistente.
\item Abrir el asistente pulsando sobre el icono.
\item Realizar la pregunta sobre el apartado "Pregunta a UBUassistant" de la interfaz cargada.
\item Si existe una única respuesta se muestra sobre la interfaz.
\imagen{Interfaz2.png}{Interfaz asistente virtual con respuesta única.}
\end{enumerate}

\subsubsection{Obtener recomendaciones}

\begin{enumerate}
\item Acceder a la página donde se encuentra el asistente.
\item Abrir el asistente pulsando sobre el icono.
\item Realizar la pregunta sobre el apartado "Pregunta a UBUassistant" de la interfaz cargada.
\item Si existen múltiples respuestas se muestran botones sobre la interfaz.
\imagen{Interfaz3.png}{Interfaz asistente virtual con respuesta múltiple.}
Al pulsar los botones nos otorga la respuesta y seleccionando la opción de no valorar la respuesta volvemos al los botones restantes por si se desea alguna otra respuesta relacionada.
\imagen{Interfaz4.png}{Interfaz asistente virtual sin respuesta.}

\item Si no existe respuesta se muestran tres recomendaciones en forma de botones en la interfaz.
\imagen{Interfaz5.png}{Interfaz asistente virtual sin respuesta.}
Al pulsarlos se muestra la respuesta oportuna.
\imagen{Interfaz6.png}{Interfaz asistente virtual sin respuesta.}

\end{enumerate}

\subsubsection{Valorar respuestas}

\begin{enumerate}
\item Acceder a la página donde se encuentra el asistente.
\item Abrir el asistente pulsando sobre el icono.
\item Realizar la pregunta sobre el apartado "Pregunta a UBUassistant" de la interfaz cargada.
\item Si existe una única respuesta se muestra el mecanismo de valoración sobre la interfaz pudiendo seleccionar de una a cinco estrellas.
\imagen{Interfaz2.png}{Mecanismo de valoración con respuesta única.}
\item Si existen múltiples respuestas se muestran botones sobre la interfaz.

Al pulsar los botones nos otorga la respuesta y nos pregunta si deseamos valorar la respuesta. 

\imagen{Interfaz7.png}{Pregunta de valoración}

Al responder de forma afirmativa nos muestra la barra de estrellas.

\imagen{Interfaz8.png}{Mecanismo de valoración con respuesta múltiple.}

Al terminar de valorar nos muestra de nuevo el panel con las recomendaciones restantes.

\imagen{Interfaz4.png}{Interfaz asistente virtual con respuesta múltiple.}

\item Si no existe respuesta se muestran tres recomendaciones en forma de botones en la interfaz.

Al pulsarlos se muestra la respuesta oportuna y el mecanismo de valoración.

\imagen{Interfaz6.png}{Mecanismo de valoración sin respuesta.}

\end{enumerate}

\subsection{Administración}

\begin{enumerate}
\item Acceder a la página principal.
\item Pulsar el botón para acceder a la página de administración.
\imagenMuyPeque{Admin1.png}{Icono de acceso a la página de administración.}
\item Autenticarse con las credenciales.
\imagen{Admin2.png}{Login página de administración.}
\end{enumerate}

\subsubsection{Listar log}

\begin{enumerate}
\item Estando autenticado como administrador.
\item Ir a la página de log.
\imagen{AdminLog.png}{Interfaz log de uso.}
\end{enumerate}

\subsubsection{Limpiar log}

\begin{enumerate}
\item Estando autenticado como administrador.
\item Ir a la página de log.
\item Pulsar sobre el botón de limpiar tabla.
\imagenMuyPeque{AdminLog1.png}{Botón limpiar tabla de log.}
\item Confirmar el borrado.
\end{enumerate}

\subsubsection{Exportar log}

\begin{enumerate}
\item Estando autenticado como administrador.
\item Ir a la página de log.
\item Pulsar sobre el botón de guardar como.
\imagenMuyPeque{AdminLog2.png}{Botón guardar como.}
\item Seleccionar el formato deseado.
\imagen{AdminLog3.png}{Formatos guardar como.}
\end{enumerate}

\subsubsection{Ordenar log}

\begin{enumerate}
\item Estando autenticado como administrador.
\item Ir a la página de log.
\item Pulsar sobre el encabezado de una columna para ordenar. Pulsar varias veces para cambiar el orden ascendente o descendente.
\end{enumerate}

\subsubsection{Listar casos}

\begin{enumerate}
\item Estando autenticado como administrador.
\item Ir a la página de modificar casos.
\imagen{AdminCases1.png}{Interfaz modificar casos.}
\end{enumerate}

\subsubsection{Añadir caso}

\begin{enumerate}
\item Estando autenticado como administrador.
\item Ir a la página de añadir caso.
\imagen{AdminCases2.png}{Interfaz añadir caso.}
\item Rellenar al menos la palabra clave 1, la categoría y la respuesta.
\item Pulsar aceptar.
\end{enumerate}

\subsubsection{Editar caso}

\begin{enumerate}
\item Estando autenticado como administrador.
\item Ir a la página de modificar casos.
\item Pulsar editar sobre el caso deseado.
\imagen{AdminCases3.png}{Interfaz editar casos.}
\item Rellenar los campos, siendo la palabra clave 1, la categoría y la respuesta obligatorios.
\item Pulsar aceptar.
\end{enumerate}

\subsubsection{Eliminar caso}

\begin{enumerate}
\item Estando autenticado como administrador.
\item Ir a la página de modificar casos.
\item Pulsar eliminar sobre el caso deseado.
\item Confirmar el borrado del caso.
\end{enumerate}

\subsubsection{Exportar casos}

\begin{enumerate}
\item Estando autenticado como administrador.
\item Ir a la página de modificar casos.
\item Pulsar sobre el botón de guardar como.
\item Seleccionar el formato deseado.
\end{enumerate}

\subsubsection{Ordenar casos}

\begin{enumerate}
\item Estando autenticado como administrador.
\item Ir a la página de modificar casos.
\item Pulsar sobre el encabezado de una columna para ordenar. Pulsar varias veces para cambiar el orden ascendente o descendente.
\end{enumerate}

\subsubsection{Listar recomendaciones}

\begin{enumerate}
\item Estando autenticado como administrador.
\item Ir a la página de aprendizaje.
\imagen{AdminLearn1.png}{Interfaz aprendizaje.}
\end{enumerate}

\subsubsection{Aprender recomendación}

\begin{enumerate}
\item Estando autenticado como administrador.
\item Ir a la página de aprendizaje.
\item Pulsar el botón aprender sobre la recomendación deseada.
\end{enumerate}

\subsubsection{Descartar recomendación}

\begin{enumerate}
\item Estando autenticado como administrador.
\item Ir a la página de aprendizaje.
\item Pulsar el botón descartar sobre la recomendación deseada.
\end{enumerate}

\subsubsection{Exportar recomendaciones}

\begin{enumerate}
\item Estando autenticado como administrador.
\item Ir a la página de aprendizaje.
\item Pulsar sobre el botón de guardar como.
\item Seleccionar el formato deseado.
\end{enumerate}

\subsubsection{Ordenar recomendaciones}

\begin{enumerate}
\item Estando autenticado como administrador.
\item Ir a la página de aprendizaje.
\item Pulsar sobre el encabezado de una columna para ordenar. Pulsar varias veces para cambiar el orden ascendente o descendente.
\end{enumerate}
