\capitulo{7}{Conclusiones y Líneas de trabajo futuras}

Se procede a detallar las conclusiones que se derivan del desarrollo del proyecto y a realizar un informe crítico indicando cómo se puede mejorar el proyecto y cómo puede continuar su desarrollo.

\section{Conclusiones}

\begin{itemize}
\tightlist
\item
Se ha conseguido realizar la idea propuesta al principio del desarrollo del proyecto. Un asistente virtual capaz de integrarse en una página web con el que poder interactuar para orientarse dentro de la web, o resolver ciertas dudas para las que ha sido preparado.
\item
El uso de un framework de inteligencia artificial como jCOLIBRI ha conseguido simplificar cierta parte del proceso de desarrollo, sin embargo, también a ocasionado otros problemas inesperados generando complejidad en aspectos que podrían no haberla tenido.
\item
Se han utilizado los conocimientos adquiridos durante la carrera, especialmente los adquiridos durante el último curso del grado, dificultando el desarrollo de la aplicación al cursar estas asignaturas de forma paralela.
\item
Se han ampliado conocimientos en lo relativo a la inteligencia artificial, desarrollo web, HTML, etc.
\item
Se han empleado ciertas herramientas que han aportado un valor añadido a la calidad del trabajo.
\item
El cambio de requisitos supuso un reto al tener que ampliar rápidamente, y en poco tiempo, conocimientos sobre técnicas que no eran conocidas.
\item
Ha sido posible completar el proyecto en el plazo establecido.
\end{itemize}

\section{Líneas de trabajo futuras}

Este proyecto puede avanzar incluyendo distintas funcionalidades:

\begin{itemize}
\tightlist
\item
Orientar la aplicación a más ámbitos dentro de la Universidad de Burgos, no solamente al guiado dentro de la propia web.
\item
Proporcionar reconocimiento de voz o comandos con los que poder interaccionar con la aplicación.
\item
Ofrecer una apariencia todavía más humana, realizando preguntas por ejemplo para refinar los resultados en caso de tener varios disponibles.
\item
Conseguir responder a preguntas de ámbito general aunque éstas no tengan relación con el ámbito para el que ha sido preparado. Por ejemplo otorgar la previsión meteorológica.
\item
Implementar la aplicación para que esté disponible en más plataformas.
\end{itemize}

Se mantuvieron conversaciones con el Servicio de Informática de la Universidad de Burgos para contemplar la posibilidad de implantar la aplicación a corto plazo y en modo pre-producción en el Servicio de Informática de la Universidad para dar asistencia a los usuarios del CAU.

Una vez definidos los roles y posibles áreas de implementación se puede ofrecer como herramienta de información al estudiante en el
"Servicio de Información y Orientación en Salud Joven Universitaria".