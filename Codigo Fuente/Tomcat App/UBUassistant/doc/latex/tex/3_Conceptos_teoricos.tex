\capitulo{3}{Conceptos teóricos}

Para la mejor compresión del funcionamiento del algoritmo de reconocimiento del texto y obtención de la respuesta mediante razonamiento basado en casos (\emph{CBR, case base reasoning}) se procede a explicar el concepto teórico en el que se ha basado su implementación.

El razonamiento basado en casos es, desde el punto de la inteligencia artificial, un sistema experto que trata de emular el comportamiento humano a la hora de intentar resolver problemas.

\section{Introducción al concepto}\label{intro-concepto}

El razonamiento basado en casos basa su funcionamiento en experiencias anteriores del propio sistema o de la persona especializada para otorgar una solución o respuesta a un caso concreto \cite{wiki:cbr}.

Las experiencias en las que se basa este razonamiento son los llamados casos. Un caso es una pieza de conocimiento que representa una experiencia y que nos sirve para alcanzar una solución. Un conjunto de casos representarán la lógica con la que trabaja un sistema experto con este razonamiento.

UBUassistant, en un primer escenario, no considera una experiencia previa para definir los casos que utilizará, sino que, sus casos están basados en la experiencia del programador.
La estructura considerada para los casos de la aplicación son cinco palabras clave o menos. Cada conjunción de estas cinco (como máximo) palabras clave forman un caso descriptivo. Cada caso descriptivo tiene asociado un caso solución. De esta manera es posible analizar un texto en busca del mayor número de palabras clave y devolver el caso solución correspondiente. El caso solución considerado para la aplicación es una cadena de texto.

Como ejemplo de caso descriptivo podemos tener la siguiente conjunción de palabras clave: [informática, ingeniería, presencial, grado, estudios]. Este caso descriptivo estará asociado con un caso solución que como ejemplo puede tomar el valor [http://www.ubu.es/grado-en-ingenieria-informatica]. De esta manera conseguimos asociar un conjunto de palabras una solución concreta.

El framework utilizado para construir un sistema CBR ha sido jCOLIBRI \cite{colibri:gen}.

\section{Funcionamiento}\label{funcionamiento}

El funcionamiento del razonamiento basado en casos consiste en un ciclo que contiene cuatro procesos diferenciados \ref{fig:CBRcycle}.

Anteriormente a este ciclo hay que realizar un paso previo, llamado preciclo (\emph{precycle)} y posteriormente a la realización del ciclo se realiza un proceso final denominado postciclo (\emph{postcycle}).

\subsection{Preciclo}\label{preciclo}

El preciclo realiza la carga de los casos desde sistema de persistencia.

En el caso del framework jCOLIBRI \cite{colibri:tut} disponemos de un método llamado preCycle() que carga nuestros casos desde nuestro sistema de persistencia MySQL.

Para poder llevar a cabo esta tarea es necesario definir el modo de acceso a los casos. Éstos pueden estar en diversos sistemas de persistencia, en nuestro caso debemos crear un conector de base de datos (\emph{DataBaseConnector}).

Una vez definido el modo de acceso a los datos debemos indicar el modo de almacenar los datos cargados en la aplicación. En nuestro caso elegimos un almacenamiento de los casos leidos en una lista (\emph{LinealCaseBase}).

La tecnología que usa internamente este framework para leer los casos de la base de datos es \emph{Hibernate}.

\subsubsection{Hibernate}\label{hibernate}

Hibernate es una herramienta para Java que permite realizar un mapeo objeto/relacional (\emph{ORM, Object Relacional Mapping}).

Con esta herramienta se consigue conectar los atributos de una clase orientada a objetos con los atributos de una base de datos relacional \cite{hibe:info}.

Hay dos formas de conseguir este mapeo, bien utilizando ficheros XML, bien usando las anotaciones de Hibernate en las clases Java. El modo de realizarlo por parte de jCOLIBRI es mediante ficheros XML.

Para ello debemos indicar al DataBaseConnector donde se encuentra el archivo XML que configura nuestra base de datos. En este fichero llamado \emph{databaseconfig.xml} se indican los nombres del resto de archivos que se encargarán de conectar con la base de datos y de establecer la estructura de los atributos de las clases que van a ser mapeadas.

En nuestro caso debemos tener una clase donde se almacenen las cinco palabras clave que forman nuestro caso de descripción, además de una clase donde se almacene el campo respuesta que forma nuestro caso solución.

\subsection{Ciclo}\label{ciclo}

Acabada la configuración y la carga de los casos, se procede a realizar el ciclo. Como ya se había comentado anteriomente un ciclo se compone de cuatro actividades.

\imagen{CBRcycle}{Ciclo del razonamiento basado en casos}

\subsubsection{Recordar}\label{recordar}

El proceso de recordar (\emph{retrieval}) consiste en recordar los casos similares al que estamos analizando.

El sistema experto, a la hora de enfrentarse a un nuevo problema, deberá seleccionar, de entre los casos disponibles, aquellos que sean más similares al del nuevo problema. De esta manera el sistema lleva a cabo un proceso similar al de una persona humana, que ante un nuevo problema que tiene que resolver rescata de su memoria acontecimientos similares que le ayuden en esa tarea.

jCOLIBRI proporciona una serie de métodos para configurar y realizar el proceso de recordar. 
En un primer paso se define la forma de evaluar la similitud. Se configuran los siguientes apartados.

\begin{itemize}
\tightlist
\item
  \textbf{Similitud global:} con este parámetro se detalla la forma de obtener un valor a partir de varios. A partir de evaluar la similitud de cada atributo, cinco atributos en nuestro caso, se obtiene un valor global que puede calcularse de varias maneras. Entre las formas de calcularlo se puede encontrar la media, el máximo, el mínimo o la distancia euclídea. El método finalmente seleccionado fue la distancia euclídea, con el que se obtivieron mejores resultados.
\item
  \textbf{Similitud de cada atributo:} define la forma de evaluar la semejanza de cada atributo. Al igual que en el anterior caso existen diferentes maneras de calcular este parámetro, seleccionando el método MaxString que se adecua a nuestra representación del caso como palabras clave.
\item
  \textbf{Peso de cada atributo:} puede existir la situación donde no se desea que la evaluación de la similitud de los atributos de un caso tengan el mimo peso en el resultado final. Por ello, el framework nos permite configurar el peso de cada atributo, en nuestro caso las cinco palabras deberán tener el mismo peso. Se ha optado por un peso de 0.75 en todos los atributos al obtener resultados satisfactorios.
\end{itemize}

Una vez configurado el modo de obtener la similitud es posible utilizar el método \emph{evaluateSimilarity} al que le pasaremos los casos, el caso obtenido del texto del usuario, y el objeto que almacena la configuración de similitud. Este método obtiene una colección donde quedan almacenados los casos junto con el resultado de la comparación con el caso generado por el texto del usuario. La solución adoptada por la aplicación desarrollada no utiliza el método de vecinos más cercanos, en su lugar se eliminan todos los resultados cuya similitud sea inferior de 0.35, de esta manera se consiguen resultados más positivos.

Por cada palabra que introduce el usuario de crea una colección con los resultados de comparar esa palabra con las palabras existentes en los casos.

\subsubsection{Reutilizar}\label{reutilizar}

El proceso de reutilización (\emph{reuse}) consiste en utilizar la información obtenida de la etapa de recordar para resolver el problema.

El framework utilizado dispone de métodos propios básicos para esta etapa del ciclo. Sin embargo, la propia documentación deja este paso abierto a los desarrolladores argumentando que es una etapa muy dependiente del ámbito de la aplicación a desarrollar.

Siguiendo estos consejos se desarrolla un algoritmo propio para obtener los resultados deseados. El algoritmo podría resumirse en los siguientes pasos:

\begin{itemize}
\tightlist
\item
  \textbf{Construir resultados finales:} Como se ha mencionado con anterioridad, en la etapa de recordar analizamos cada palabra introducida por el usuario con las palabras de los distintos casos generando una colección con estos resultados (eliminando posteriormente los resultados con una evaluación peor que 0.35). Con esto conseguimos obtener las mejores soluciones o respuestas para cada palabra, si varias palabras comparten la misma solución (forman parte del mismo caso descriptivo) se combinan.
  
  Se expone un ejemplo para su mejor comprensión:
  Ante una búsqueda por parte del usuario de las palabras 
  [ingeniería, informática, becas e internacionales] 
  el algoritmo generaría los siguientes resultados finales, 
  [ingeniería, informática] \textrightarrow [http://www.ubu.es/grado-en-ingenieria-informatica] y 
  [becas, internacionales] \textrightarrow [http://www.ubu.es/becas-de-cooperacion]
\item
  \textbf{Obtención de solución única:} Si en el proceso de construir resultados finales obtenemos solamente una respuesta, ésta es la que será mostrada como solución.
  \item
  \textbf{Obtención de solución múltiple:} En el caso de que el proceso de construir resultados finales obtenga varios resultados, como el ejemplo expuesto, la aplicación mostrará todas las opciones para que el usuario elija la que desee.
    \item
  \textbf{No existencia de resultados:} Se puede dar el caso más que probable de que el usuario realice una búsqueda con palabras que no existen en nuestros casos, en esta situación se mostrarán tres sugerencias donde se encontrarán los casos con mejor similitud aunque no lleguen a la anterior restricción de 0.35.
\end{itemize}

\subsubsection{Revisar}\label{Revisar}

El proceso de reutilizar nos otorga una solución, sin embargo, es necesario una etapa de revisión (\emph{revise}).

Esta etapa consiste en determinar si una respuesta puede considerarse como válida en términos de solucionar el problema planteado.

De nuevo, como en ocasiones anteriores nos encontramos con una fase muy dependiente de la aplicación desarrollada, por lo que jCOLIBRI nos anima a utilizar nuestro propio sistema de revisión en aras de obtener mejores resultados.

El sistema de revisión propuesto no es automático sino que depende del usuario de la aplicación. Será el usuario el que decida si una respuesta otorgada se ajusta a sus pretensiones. Para facilitar el sistema de valoración de respuestas se ha decidido otorgar al usuario un sistema basado en estrellas, siendo una estrella la valoración más baja y cinco estrellas el valor más alto.

Las valoraciones de los usuarios quedan almacenadas en una tabla de log de uso que puede ser visionada por el administrador de la aplicación desde la página destinada a la administración de la aplicación. Quedará a juicio del administrador cambiar la respuesta de un determinado caso que haya obtenido de manera continuada valoraciones negativas.

\subsubsection{Retener}\label{retener}

Retener (\emph{retain}) es la etapa donde nuevos casos que pueden ser útiles en el futuro son almacenados.

Se opta por realizar un algoritmo independiente del framework que se adecúe más a nuestro problema.

La implementación de la etapa de retención llevada a cabo consiste en almacenar en una tabla de la base de datos una nueva palabra clave junto con una respuesta asociada. Cuando un usuario introduce una búsqueda para la que no existe una respuesta se muestran tres recomendaciones de búsqueda. El usuario puede obviar esas recomendaciones y proceder a realizar una nueva búsqueda, pero, en el caso de que acepte una de las recomendaciones, la respuesta de esta recomendación será asociada con la palabra buscada que ha generado esta recomendación.

Para aclarar el concepto se explica un ejemplo. Si un usuario busca una frase que contiene la palabra [mechanic] junto con otras palabras que tampoco tienen respuesta, se le otorgará una sugerencia de búsqueda que será [mecánica]. Si el usuario acepta la recomendación se almacenará en una tabla la palabra [mechanic] junto con la solución de la palabra [mecánica].

Este proceso de retener también requiere de un administrador que decida las palabras que finalmente se añadirán como un nuevo caso, ya que realizarlo de manera automática puede provocar un aprendizaje erróneo.


\subsection{Postciclo}\label{postciclo}

El postciclo es la última etapa después de realizar el ciclo. Se encargará de cerrar la conexión con la base de datos que carga los casos.

En nuestra aplicación podemos realizar el postciclo, mediante el método \emph{postCycle}, en cualquier momento después de la etapa de reutilizar, ya que el algoritmo de las siguientes fases es independiente de la conexión con las tablas de casos de la base de datos.