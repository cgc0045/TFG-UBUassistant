\capitulo{4}{Técnicas y herramientas}

\section{Patrones de diseño}
\subsection{Modelo Vista Controlador}

Modelo vista controlador (\emph{MVC}) es un patrón arquitectónico que nos ayuda a separar los datos, la lógica de negocio y la interfaz de usuario \cite{mvc:wiki}.

\imagen{MVC}{Diagrama Modelo Vista Controlador}

El significado de cada uno de los tres componentes de este modelo son:

\begin{itemize}
	\tightlist
	\item
	\textbf{Modelo:} representa los datos que utiliza la aplicación.
	\item
	\textbf{Vista:} Muestra al usuario la información.
	\item
	\textbf{Controlador:} Controla las interacciones con el usuario.
\end{itemize}


Estos conceptos pueden ser aplicados a nuestra aplicación Android, obteniendo así un Modelo Vista Controlador muy bien diferenciado y estructurado. Esto lo podemos ver claramente en el siguiente diagrama.
\imagen{MVC-Android}{Modelo Vista Controlador en Android}

\section{Control de versiones}

\begin{itemize}
	\tightlist
	\item
	Herramientas consideradas: \href{https://git-scm.com/}{Git}.
	\item
	Herramienta elegida: \href{https://git-scm.com/}{Git}.
\end{itemize}

Git es un software de control de versiones pensado para proyectos que poseen una gran cantidad de ficheros fuente cuyo propósito es registrar todos los cambios efectuados en dichos ficheros. Además es un software de código libre distribuido bajo licencia \textit{GPL GNU} \cite{git:wiki}.

\section{Hosting del repositorio}

\begin{itemize}
	\tightlist
	\item
	Plataformas consideradas: \href{https://bitbucket.org/}{Bitbucket} y \href{https://github.com/}{Github}.
	\item
	Plataforma elegida: \href{https://github.com/}{Github}.
\end{itemize}

GitHub es una plataforma que es utilizada para alojar proyectos, los cuales emplean Git como sistema de control de versiones.

GitHub ha sido elegido frente a BitBucket debido a que es una plataforma que se ha ido utilizando en diferentes asignaturas como Gestión de Proyectos.

\section{Gestión del repositorio}

\begin{itemize}
	\tightlist
	\item
	Herramientas consideradas: \href{https://www.gitkraken.com/}{GitKraken} y \href{https://desktop.github.com/}{GitDesktop}.
	\item
	Herramienta elegida: \href{https://www.gitkraken.com/}{GitKraken}.
\end{itemize}

GitKraken es una aplicación que sirve para gestionar de una forma más sencilla nuestro repositorio de GitHub. Es una herramienta multiplataforma compatible con Windows, Mac y Linux.

La decisión de usar GitKraken frente a GitDesktop está basada en la experiencia personal, ya que previamente había trabajado con ambas herramientas. A parte de esto, la decisión de usar GitKraken está fundamentada en su compatibilidad con sistemas Linux.

\section{Sistema Operativo}

\begin{itemize}
	\tightlist
	\item
	Sistemas considerados: \href{https://www.microsoft.com/es-es/windows}{Windows 10}, \href{https://www.ubuntu.com/}{Ubuntu} y \href{https://linuxmint.com/}{Linux Mint}.
	\item
	Sistema elegido: \href{https://linuxmint.com/}{Linux Mint}.
\end{itemize}

Linux Mint es un sistema operativo Linux que utiliza un núcleo de sistema basado en Debian y Ubuntu.

En un primer momento se planteó si usar Windows o Linux como sistema operativo. Finalmente se decidió usar Linux ya que se consideró que la gestión de recursos la hace de forma más eficiente y emplea menos recursos. Una vez tomada la decisión de usar Linux, se barajaron dos opciones, si usar Ubuntu o Linux Mint. La decisión se tomo en base a la experiencia personal, ya que se había trabajado con anterioridad con ambos sistemas operativos.

\section{Entorno de desarrollo integrado (IDE) para JAVA}

\begin{itemize}
	\tightlist
	\item
	IDE's considerados: \href{https://www.jetbrains.com/idea/}{IntelliJ}, \href{https://netbeans.org/}{NetBeans} y \href{https://eclipse.org/}{Eclipse}.
	\item
	IDE elegidos: \href{https://eclipse.org/}{Eclipse}.
\end{itemize}

Eclipse es un software que está formado por diferentes herramientas de programación de código abierto multiplataforma y se encuentra bajo la licencia de software libre \textit{Eclipse Public License}. Para el desarrollo de este proceso, se ha empleado la versión EE, la cual permite realizar páginas web basadas en Java en formato JSP \cite{eclipse:info}.

De todos los softwares valorados para emplear como IDE se decidió utilizar Eclipse ya que es una herramienta que resulta familiar debido a que ha sido utilizada en todas aquellas asignaturas del Grado en el que se ha utilizado el lenguaje de programación JAVA.


\section{Entorno de desarrollo integrado (IDE) para Android}

\begin{itemize}
	\tightlist
	\item
	IDE's considerados: \href{https://developer.android.com/studio/}{Android Studio}, \href{http://www.android-ide.com/}{AIDE} y \href{https://eclipse.org/}{Eclipse}.
	\item
	IDE elegidos: \href{https://developer.android.com/studio/}{Android Studio}.
\end{itemize}

Android Studio es el IDE oficial para el desarrollo de aplicaciones para Android. Este IDE es el sucesor de Eclipse como IDE oficial de desarrollo, funciona gratuitamente bajo \textit{Licencia Apache 2.0} y está disponible para Windows, macOS y Linux \cite{androidstudio:wiki}.

Esta herramienta ha sido elegida principalmente por que es la herramienta oficial que nos ofrece Google y trae consigo todo lo necesario para desarrollar la aplicación, sin tener que descargar complementos para el software, como pasa en el caso de Eclipse.

\section{Documentación}

\begin{itemize}
	\tightlist
	\item
	Herramientas consideradas: \href{http://www.xm1math.net/texmaker/}{TexMaker}, \href{https://www.texstudio.org/}{TexStudio}, \href{https://es.sharelatex.com/}{ShareLaTex} y \href{https://www.openoffice.org/es/}{OpenOffice}.
	\item
	Herramienta elegida: \href{http://www.xm1math.net/texmaker/}{TexMaker}.
\end{itemize}

Entre realizar la documentación con LaTex u OpenOffice se decidió emplear LaTex, ya que los resultados que se obtienen son una documentación obtenida con una gran calidad tipográfica.

LaTex es un sistema empleado para la composición de textos escritos con una alta calidad tipográfica tales como artículos académicos, tesis y libros técnicos. Es un software libre bajo la licencia \textit{LPPL} \cite{latex:wiki}.

TexStudio es un editor para LaTex de código abierto y multiplataforma que es una evolución del IDE TexMaker. Este IDE esta bajo la licencia \textit{GNU GPL} \cite{texstudio:wiki}.

La elección de este editor es que es offline, por lo que no dependemos de una conexión a Internet, y que es una evolución de TexMaker, por lo que funcionalidades añadidas respecto a este IDE.
