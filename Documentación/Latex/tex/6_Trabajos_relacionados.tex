\capitulo{6}{Trabajos relacionados}

Actualmente nos podemos encontrar con diversas técnicas para crear un asistente virtual. Una de las técnicas que nos podemos encontrar es el uso de inteligencia artificial. Dentro de la inteligencia artificial, una de las técnicas empleadas en estos casos es la de razonamiento basado en casos.

\section{Artículos}
\subsection{Chatter-bots: Making AI work for your business}
Este artículo escrito, proveniente de una página especializada en el razonamiento basado en casos, escrito por Paul White, nos comenta cuales son los pasos que debe de seguir nuestro asistente virtual para que funcione de forma correcta. Los consejos que nos aporta es que nuestro asistente virtual tiene que tener los objetivos muy claros, que al principio empiece con unos conocimientos básicos y que los conocimientos vayan ampliándose progresivamente a lo largo del tiempo.

\section{Proyectos}
\subsection{TFG - UBUassistant de Daniel Santidrián Alonso}
Este TFG es la base de este proyecto, ya que de él está heredado parte del código, sobre todo a aquello a lo referente al sistema de razonamiento basado en casos, utilizando \emph{jCOLIBRI 2} \cite{tfg:git}.

\subsection{Cómo crear una aplicación de chat para Android usando Firebase}
Este tutorial, realizado por Ashraff Hathibelagal, es la base en la cual se ha basado la aplicación Android. Principalmente se ha empleado como base el diseño de la aplicación, así como la forma de mostrar los mensajes, aunque posteriormente fue modificada en busca de una estética más propicia para el tipo de aplicación \cite{chat:tut}.

\section{Fortalezas y debilidades del proyecto}
\subsection{Fortalezas del proyecto}

Las fortalezas que posee este proyecto son:
\begin{itemize}
	\tightlist
	\item
	Aplicación Android compatible con todos los smartphones que posean versión 4.4 o posterior.
	\item 
	Interfaz de usuario sencilla e intuitiva.
	\item 
	Posibilidad de utilizar voz para realizar las búsquedas.
	\item 
	Facilidad de creación de una interfaz de cliente utilizando otros lenguajes de programación.
	\item 
	Bajo consumo de datos para realizar las consultas.
	\item 
	Servidor alojado en la nube, por lo que la aplicación se puede usar desde cualquier localización.
\end{itemize}

\subsection{Debilidades del proyecto}

Por el contrario, existen algunas debilidades dentro del proyecto:

\begin{itemize}
	\tightlist
	\item
	La aplicación Android es solo para los usuarios, por lo que hay que seguir empleando la web para administrar el aprendizaje.
	\item 
	La aplicación siempre necesita conexión a Internet para poder funcionar.
\end{itemize}
