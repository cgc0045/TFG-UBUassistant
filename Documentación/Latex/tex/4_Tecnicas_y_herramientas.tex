\capitulo{4}{Técnicas y herramientas}

\section{Patrones de diseño}
\subsection{Modelo Vista Controlador}

Modelo vista controlador (\emph{MVC}) es un patrón arquitectónico que nos ayuda a separar los datos, la lógica de negocio y la interfaz de usuario \cite{mvc:wiki}.

\imagen{MVC}{Diagrama Modelo Vista Controlador}

El significado de cada uno de los tres componentes de este modelo son:

\begin{itemize}
	\tightlist
	\item
	\textbf{Modelo:} representa los datos que utiliza la aplicación.
	\item
	\textbf{Vista:} Muestra al usuario la información.
	\item
	\textbf{Controlador:} Controla las interacciones con el usuario.
\end{itemize}


Estos conceptos pueden ser aplicados a nuestra aplicación Android, obteniendo así un Modelo Vista Controlador muy bien diferenciado y estructurado. Esto lo podemos ver claramente en el siguiente diagrama.
\imagen{MVC-Android}{Modelo Vista Controlador en Android}

\section{Control de versiones}

\begin{itemize}
	\tightlist
	\item
	Herramientas consideradas: \href{https://git-scm.com/}{Git}.
	\item
	Herramienta elegida: \href{https://git-scm.com/}{Git}.
\end{itemize}

Git es un software de control de versiones pensado para proyectos que poseen una gran cantidad de ficheros fuente cuyo propósito es registrar todos los cambios efectuados en dichos ficheros. Además es un software de código libre distribuido bajo licencia \textit{GPL GNU} \cite{git:wiki}.

\section{Hosting del repositorio}

\begin{itemize}
	\tightlist
	\item
	Plataformas consideradas: \href{https://bitbucket.org/}{Bitbucket} y \href{https://github.com/}{Github}.
	\item
	Plataforma elegida: \href{https://github.com/}{Github}.
\end{itemize}

GitHub es una plataforma que es utilizada para alojar proyectos, los cuales emplean Git como sistema de control de versiones.

GitHub ha sido elegido frente a BitBucket debido a que es una plataforma que se ha ido utilizando en diferentes asignaturas como Gestión de Proyectos.

\section{Gestión del repositorio}

\begin{itemize}
	\tightlist
	\item
	Herramientas consideradas: \href{https://www.gitkraken.com/}{GitKraken} y \href{https://desktop.github.com/}{GitDesktop}.
	\item
	Herramienta elegida: \href{https://www.gitkraken.com/}{GitKraken}.
\end{itemize}

GitKraken es una aplicación que sirve para gestionar de una forma más sencilla nuestro repositorio de GitHub. Es una herramienta multiplataforma compatible con Windows, Mac y Linux.

La decisión de usar GitKraken frente a GitDesktop está basada en la experiencia personal, ya que previamente había trabajado con ambas herramientas. A parte de esto, la decisión de usar GitKraken está fundamentada en su compatibilidad con sistemas Linux.

\section{Sistema Operativo}

\begin{itemize}
	\tightlist
	\item
	Sistemas considerados: \href{https://www.microsoft.com/es-es/windows}{Windows 10}, \href{https://www.ubuntu.com/}{Ubuntu} y \href{https://linuxmint.com/}{Linux Mint}.
	\item
	Sistema elegido: \href{https://linuxmint.com/}{Linux Mint}.
\end{itemize}

Linux Mint es un sistema operativo Linux que utiliza un núcleo de sistema basado en Debian y Ubuntu.

En un primer momento se planteó si usar Windows o Linux como sistema operativo. Finalmente se decidió usar Linux ya que se consideró que la gestión de recursos la hace de forma más eficiente y emplea menos recursos. Una vez tomada la decisión de usar Linux, se barajaron dos opciones, si usar Ubuntu o Linux Mint. La decisión se tomo en base a la experiencia personal, ya que se había trabajado con anterioridad con ambos sistemas operativos.