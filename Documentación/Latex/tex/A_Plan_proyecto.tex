\apendice{Plan de Proyecto Software}

\section{Introducción}
Una correcta planificación del proyecto es muy importante. Esto se debe por que de ello dependerá de que nuestro proyecto vaya evolucionando de una forma correcta y que se cumplan los tiempos de entrega, evitando así todos los retrasos posibles. Esta fase de planificación está compuesta por dos apartados:

\begin{itemize}
	\tightlist
	\item
	\textbf{Planificación temporal:} esta planificación consta principalmente en establecer los diferentes tiempos del proyecto. En este apartado se establece la fecha de inicio y la fecha de finalización del proyecto. A parte se establece la duración de cada una de las fases del proyecto, aunque estas pueden adaptarse algo a lo largo del proyecto en función del trabajo realizado a lo largo de cada uno de los sprints.
	
	\item
	\textbf{Estudio de viabilidad:} este estudio es muy importante, ya que es lo que nos va a permitir valorar si el proyecto nos va a permitir continuar hacia delante con él o no. Dentro de este estudio nos encontramos con dos fases;
	\begin{itemize}
		\tightlist
		\item
		\textbf{Viabilidad económica:} es el estudio en el que vamos a estudiar los costes del proyecto, tanto de tiempo, como de personal así como los recursos de terceros que son necesarios para realizarlo.
		\item
		\textbf{Viabilidad legal:} este estudio es el más importante de todos, ya que es el que necesitamos para saber si el proyecto que vamos a hacer está permitido por ley. A parte de esto, también se estudia todo lo relacionado con la adquisición de las diferentes licencias que sean necesarias para el desarrollo del proyecto.
	\end{itemize}
\end{itemize}

\section{Planificación temporal}
Para el desarrollo del proyecto se optó por seguir la metodología de desarrollo ágil \textit{Scrum} \cite{scrum:wiki}.

Debido al tamaño del proyecto realizado, no se ha podido seguir a la totalidad los rasgos de esta metodología, pero se ha intentado adaptar todo lo posible, manteniendo la esencia de la misma. Las diferentes pausas que se siguieron fueron las siguientes:

\begin{itemize}
	\tightlist
	\item
	Desarrollo incremental de la aplicación. Para ello se realizó a través de sprints.
	\item 
	Los sprints se realizaron de una duración de una semana cada uno.
	\item 
	Reuniones al final de cada uno de los sprint, en los cuales se valoraba el trabajo realizado en el sprint y se establecían los objetivos para el próximo sprint.
\end{itemize}

Las tareas realizadas en cada uno de los sprints se comenta a continuación.

\subsection{Sprint 0 (31/02/2018 - 06/03/2018)}

En esta primera reunión lo que se hizo fue hablar sobre la propuesta del tutor para realizar el TFG. En ella se me comentó en que consistió el TFG que me ha tocado mejorar, como un poco las ideas que se quería para mejorar el citado TFG.

La tarea que se me encomendó en este primer sprint fue el de que hiciera una breve investigación sobre el funcionamiento de los algoritmos de razonamiento basado en casos.

\subsection{Sprint 1 (07/03/2018 - 13/03/2018)}

En esta reunión el tutor me ofreció todo el material relacionado con el TFG a mejorar, así como una definición formal de los requisitos a cumplir durante el desarrollo del proyecto.

El objetivo que se estableció en este sprint es el de leerme la documentación del TFG e ir empezando la comprensión del proyecto para ir recogiendo todas las dudas posibles para que pudieran ser respondidas las dudas en la próxima reunión.

\subsection{Sprint 2 (14/03/2018 - 20/03/2018)}

En esta reunión, lo primero que se hizo, fue solucionar todas aquellas dudas que fueron surgiendo a lo largo del sprint anterior y se establecieron los objetivos para el próximo sprint.

El objetivo de este sprint fue el de analizar todas aquellas herramientas que creyera que fueran necesarias para el desarrollo del proyecto y dejar constancia de porque la decisión de emplear esas herramientas en vez de usar otras herramientas.

\subsection{Sprint 3 (21/03/2018 - 27/03/2018)}

Lo primero que se realizó fue valorar las herramientas elegidas para el proyecto, así como alguna recomendación de otras herramientas por parte del tutor, dejando la valoración de esas herramientas para el próximo sprint.

A parte de ese objetivo, también se estableció como objetivo investigar la forma de implementar la introducción de texto mediante voz en nuestra aplicación de Android.

\subsection{Sprint 4 (28/03/2018 - 10/04/2018)}

Al comienzo de la reunión se terminó de valorar las herramientas sugeridas por el profesor en la reunión anterior. Además también se valoró la decisión tomada respecto al reconocimiento de voz dentro de la aplicación, tomando como buena la solución encontrada para ello.

Debido a que este sprint duró dos semanas a raíz de que la Universidad de Burgos cerraba por vacaciones de Semana Santa se establecieron más objetivos para cubrir trabajo en las dos semanas. El primer objetivo que se estableció fue el de investigar la viabilidad de realizar el cambio de framework de razonamiento basado en casos, ya que es uno de los objetivos del proyecto. A parte de esto, también se estableció como objetivo que fuera 
creando el repositorio en GitHub y que se fuera importando el anterior TFG para poder empezar a realizar cambios en el código.






















\section{Estudio de viabilidad}

\subsection{Viabilidad económica}

\subsection{Viabilidad legal}


