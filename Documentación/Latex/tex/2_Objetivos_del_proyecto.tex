\capitulo{2}{Objetivos del proyecto}

En este apartado se explica cuáles han sido los objetivos que se han perseguido durante el proyecto.

\section{Objetivos software}\label{objetivos-software}

\begin{itemize}
	\tightlist
	\item
	Desarrollar una aplicación Android que permita a cualquier usuario poder acceder de forma sencilla rápida a la diferente información que contiene la página de la Universidad de Burgos utilizando su teléfono móvil.
	\item
	Mejorar la experiencia de usuario a través de su dispositivo móvil.
	\item
	Realizar un aprendizaje de nuevos casos en los que no existe respuesta de manera supervisada.
\end{itemize}

\section{Objetivos técnicos}\label{objetivos-tecnicos}

\begin{itemize}
	\tightlist
	\item
	Aplicar Scrum, en la medida de lo posible, como metodología de desarrollo ágil.
	\item
	Realizar la aplicación siguiendo el concepto de Modelo Vista Controlador, separando la interfaz de usuario, el motor de la aplicación y los datos.
	\item
	Obtener las respuestas del motor de la aplicación empleando un estándar, haciendo así que la aplicación sea más modular y multiplataforma.
	\item
	Acceder a una base de datos MySQL mediante Hibernate y JDBC.
	\item
	Servirse de GitHub como sistema de control de versiones.
	\item
	Utilizar herramientas de control de calidad del software como SonarQube, RefactorIt o InCode.
	
\end{itemize}

\section{Objetivos personales}\label{objetivos-personales}

\begin{itemize}
	\tightlist
	\item
	Emplear la mayor cantidad posible de conocimientos adquiridos durante la carrera.
	\item
	Ampliar los conocimientos adquiridos durante la carrera utilizando nuevos lenguajes de programación (Android).
	\item 
	Conseguir una aplicación que facilite las tareas de búsqueda en la página de la Universidad de Burgos.
\end{itemize}
